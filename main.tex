\documentclass[a4paper]{article}
\usepackage[utf8]{inputenc}
\usepackage{graphicx}
\usepackage{float}
\usepackage{hyperref}
\usepackage{xcolor}

\title{Summary of Blockchain and IoT-based
architecture design for
intellectual property protection Research Paper}

\date{21011101131 AI-DS B}
\author{Surya Narayan}
\begin{document}
\maketitle

\hfill\href{https://www.emerald.com/insight/content/doi/10.1108/IJCS-03-2020-0007/full/html}{Link to the Orginal Paper}
\section{Summary}
This paper proposes a high-level architecture design of \textbf{blockchain} and \textbf{IoT}-based \href{https://en.wikipedia.org/wiki/Intellectual_property}{\textcolor{gray}{intellectual property}} protection system, which can help to process three types of intellectual property: 
\begin{itemize}
  \item patents, copyrights, trademarks etc.,
  \item industrial design, trade dress, craft works,
  \item plant variety rights
\end{itemize}
Using blockchain peer-to-peer network and IoT devices the proposed method can help people to establish a trusted self organized open and ecological IP protection system.\\ 


\section{Key Contributions }\hfill(with ref. to author's approach)\\
\\
Using Blockchain P2P network and IoT devices , the system allows for the secure, transparent, trustable, traceable, auditable and tamper-proof recording of transactions related to intellectual property protection. The data can be transferred between digital devices without any human interaction or manual input from either humans or computers.\\
\\
The proposed system uses IoT devices to link computing devices and digitized machines, things, objects, animals and people that are provided with digital unique identifiers (UIDs).

\subsection{Architecture}
An  \href{https://www.researchgate.net/publication/317585066_Integration_of_Cloud_Computing_with_Internet_of_Things_Challenges_and_Open_Issues}{\textcolor{gray}{architecture}} which represents the communication between IoT devices based on their UID's provided
is given below. Layers used in this architecture:
\begin{itemize}
    \item Sensing Layer
    \item Network Layer
    \item Application Layer
\end{itemize}
\begin{figure}[H]
  \includegraphics[width=\linewidth]{image3.png}
  \caption{Cloud-based IoT architecture}
  \label{fig:Blockchain and IoT system architecture}
\end{figure}

\begin{figure}[H]
  \includegraphics[width=\linewidth]{image1.png}
  \caption{Blockchain architecture proposed by the author}
  \label{fig:Blockchain architecture proposed by the author}
\end{figure}
Blockchain and IoT system architecture with the necessary Iot protocols being specified in the figure given below.
\begin{figure}[H]
  \includegraphics[width=\linewidth]{image2.png}
  \caption{Blockchain and IoT architecture }
  \label{fig:Blockchain and IoT system architecture}
\end{figure}


\section{Conclusion}
Blockchain and IoT technologies can help us to build a trusted, self-organized, open and ecological IP protection system, which can involve all different parties in the IP protection and trade procedures, and even they may not trust each other.


\end{document}

